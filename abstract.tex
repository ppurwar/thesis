%% ----------------------------------------------------------------------------
% BIWI SA/MA thesis template
%
% Created 09/29/2006 by Andreas Ess
% Extended 13/02/2009 by Jan Lesniak - jlesniak@vision.ee.ethz.ch
%% ----------------------------------------------------------------------------

\newpage
\vspace{3cm}

\chapter*{Abstract}

Image segmentation is a fundamental middle-level computer vision task, necessary to higher level image understanding, such as semantic image analysis, scene understanding, diseases diagnosis, etc. Recently, convolutional neural networks (CNN) have set new state-of-the-art standards in the field, and attract a lot of attention among both practitioners and experts. Beyond the accuracy they can achieve in many computer vision tasks, their attractivity lies in their versatility, their capability to be reused and transferred to similar problems, their layered architecture, and their capability to learn meaningful features. Nonetheless, in practice, the main obstacle is to obtain a sufficient number of annotated image data for the task at hand. \par
This poses a major problem for application scenarios where large annotated data-sets are not available or difficult to obtain. This thesis tackles such a case, and is motivated by the problems faced in an imaging facility: annotations can be difficult, even for experts, images are very diverse in nature, and in appearance. \par
In this thesis, we tackle the problem of applying CNNs in a semi-supervised and semi-interactive image segmentation scenario. We study how the segmentation accuracy of a segmentation pipeline evolves with the annotation effort of the user. Our base CNN is OSVOS, developed for video segmentation, where only a very small subset needs to be annotated (one to three frames fully labeled). This work focuses on partially annotated images with scribbles. We compare two strategies: a random forest (RF)-based and CNN-based. We derive a loss function for training CNNs from scribbles. Varying amounts and different types of scribbles are used to train either a RF or our modified OSVOS. We show that both the quantity and quality of the annotations are important for increasing the segmentation accuracy, and that the RF-based pipeline is better for the low-annotation regime. \par
We also compare different post-processing strategies of the predicted soft segmentation mask: thresholding and variational image segmentation. We show that the type of labeling cost used in the variational model matters. The model we propose ensures that one can always benefit from post-processing the soft-mask with a variational method. This is not the case for the widespread cost function in the literature, that can degrade the segmentation accuracy, even when the
RF or the CNN make good predictions.
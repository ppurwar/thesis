%% ----------------------------------------------------------------------------
% BIWI SA/MA thesis template
%
% Created 09/29/2006 by Andreas Ess
% Extended 13/02/2009 by Jan Lesniak - jlesniak@vision.ee.ethz.ch
%% ----------------------------------------------------------------------------

\chapter{Conclusion}
The target of the project to estimate PSF using sparse convex combination of Gaussian kernels is obtained by formulation as Bayesian inverse problem. The optimization problem solved using ADMM and \dic of Gaussians gives satisfactory results with a bound on number of iterations. The value of cost function drops rapidly from first iteration itself. Therefore, running a limited number of iteration gives better results for ADMM in comparison to \cma. Thus, use of ADMM helps us to obtain a good approximation efficiently. The algorithm gives an accurate estimation in case of synthetic data generated using Gaussians. The results obtained for non-gaussian PSF (Gibson and Lanni 3D optical model) are also satisfactory. The PSF obtained is able to follow the shape of complex true PSF with a limited number of Gaussians. The kernels in the \dic can be ranked according to weights and can be selected as per our requirement.  \par
There are few drawbacks in our algorithm that requires modification. The first one is effect of biasing due to shrinkage operator in $\vv_2$-subproblem for high values of $\lambda$. This requires post-processing of results to remove this biasing effect. The other observation is about use of $L$-1 norm for obtaining sparse solutions in case of optimizing for $\phi$ and \w separately. It was observed that for our optimization problem, there is no effect of increase of $\lambda$ after a certain point. The use of $L$-1 norm cease to increase sparsity with increase of parameter, $\lambda$ . The relaxation of cost term, $\mathbf{card}(\w)$, to $L$-1 norm does not work with simplex constraint on \w. Therefore, we have to use a different definition for obtaining a sparse solution in simplex space \cite{bach}.\par
Apart from problem formulation and algorithm used, the most crucial step for getting good results is choice of \dic i.e means and variances of Gaussian kernels. The algorithm is not able to give good results, if the \dic is badly chosen. The choice of means of \dic by selecting brighter pixels works well, while, the variances of Gaussians are tested and tried to build a \dic for giving satisfactory results. It can be observed that choice of variances is not as direct as choice of means. We propose to use Gaussian scale space for getting an estimate on value of variances to be used. The other way is to learn dictionary with results from optimization \cite{sls}.\par
Overall, the project gives you insight in image formation process for microscope using fluoroscent beads. The project enables us to formulate a mathematical optimization problem using Bayesian inverse problem for a physical process of fluoroscent microscopy. It also introduces us to different convex optimization algorithms and their advantages.

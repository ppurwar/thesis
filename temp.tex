

Therefore, we plan to use an approach that can be modified easily to new images/objects to be segmented. This influences us to use hybrid of data and model driven approach. We decided to use an semi-supervised approach using Maximum-aposteriori inference. The idea is to propagate segmentation label from labeled to unlabeled pixels using an prediction model and a prior, as shown in figure 1.1. The prediction model and prior are combined using Variational image processing methods. As shown in figure 1.1, this can be used to generate segmentation or can be used to train other prediction model if needed.
We can observe that the limitation on availability of fully annotated ground truth images inspires us to observe the trade-off between partially labeled data and accuracy of segmented images. This will enable the experts to use their annotation budget i.e. the pixels labeled by experts for training, efficiently giving best results. In addition, we observe effect of different ways of introducing maximum-likelihood and different regularisation parameter for total variation.

Search for 
cite
appendix
figure
add_image